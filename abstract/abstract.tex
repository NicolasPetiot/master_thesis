\thispagestyle{empty}% pour une page sans en-tête ni pieds de page
%\chapter*{Résumés}
\addcontentsline{toc}{chapter}{Abstract}%Pour l'ajout dans la table des matières au même rang que chapitre
\begin{abstract}
Proteins are complex biomolecules that are critical to the functioning of living organisms. They are made up of sequences of amino-acids that fold into specific three-dimentional structures, which is highly related to their function. The process of protein folding is considered one of the most challenging problems in the field of biology and biochemistry, as it involves a delicate interplay of chemical and physical forces that determine the final shape of the protein. AlphaFold is a groundbreaking tool developped by DeepMind that uses artificial intelligence algorithms to pedict 3D structure of proteins based on their amino-acid sequence. This tool has the potential to revolutionize the study of enzymes as it provides fast and accurate way to predict the structure of molecules that have catalytic properties. The present work aim at using AlphaFold to study the catalytic properties of Glutathione Transferase (GST), especially from class $\delta$ and $\varepsilon$ of \textit{drosophilia melanogaster}, whith the ultimate goal of being able to design brand new sequences for enzymes with improved catalytic efficiancy.\\
\noindent In addition to using AlphaFold for the generation of 3D structures, molecular dynamics simulations can also be performed based on these structures. It allows to predict some of their potential behaviour and study various factors that may influence the function of the protein, such as thermal fluctuations or protein-ligand interactions. Simulationg these processes gives a deeper understanding of how proteins function and allows to identify areas for further investigation and improvements. 
	  
\end{abstract}
\thispagestyle{empty}%