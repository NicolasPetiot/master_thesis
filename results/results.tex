\chapter{Results and Discussion}

\section{Sequences and Static Structures}

\begin{figure}[h!]
	\label{Sequence & Structure}
	%\raggedright\textbf{A}
	%\includegraphics[width = .99\linewidth]{figures/MSA_1-120_empty.pdf}\\[.5cm] % MSA
	\begin{minipage}{.32\linewidth}
		\textbf{A}\\
		\includegraphics[height = 5cm]{figures/PercentID_matrix.jpg} % ID matrix
	\end{minipage}
	\begin{minipage}{.32\linewidth}
		\textbf{B}\\
		\includegraphics[height = 5cm]{figures/GSTD1_GSHs.png} % 40 positions of GSH		
	\end{minipage}
	\begin{minipage}{.32\linewidth}
		\textbf{C}\\
		\includegraphics[height = 5cm]{figures/RMSD_matrix.jpg} % RMSD matrix
	\end{minipage}
	\caption{Sequences and structures associated to the selected \textit{Drosophila Melanogaster}'s GSTs}
\end{figure}

\noindent The first step of our methodology was to study the sequences of the GSTs of interest. We computed their Multiple Sequence Alignment as well as the percent identity matrix associated (see Fig. \ref{Sequence & Structure} pannel A). This allows to identify regions of high/low conservation as well as clusters of identity. For instance, it is clearly visible that among the class $\delta$, the GSTs are self similar from a sequence point of view. In contradiction, it seems that among the class $\varepsilon$, the GSTs E5, E6, E7, E8 are self similar but the other ones seems much more different. The sequences were then used as a base for the AlphaFold program to predict the 3D structure (pannel B) and pairs of structures were compared using the Root Mean Squared Deviation (i.e. geometrical distances between atoms, pannel C) and once again in the associated matrix, one can identify two main clusters for class $\delta$ and $\varepsilon$ but here the E10 gives much higher values than we might have expect. This is due to a much longer sequence in the terminal part that make the RMSD higher.

\noindent The information about the structures can be completed by an information about the position of the Glutathione. As mantioned earlier, the program AlphaFill allows to make such precisions and this gives $40$ different positions of Glutathion-like ligands in the GSTD1 (i.e. ligands that are chemically close to Glutathione), those positions are represented in the 3D structure (pannel B). Distances between atoms allowed us to determine from these data the residues that are involved in the protein-ligand binding as well as the dimerization of the structure. Indeed, two atoms that are closer than $3$\AA ~were considered as in contact. This information can be computed for all 25 structures and projected on the MSA matrix computed before. This gives the following representation (see Fig. \ref{MSA + AF + AFi}), where we computed the probability of a given residue to be in the binding site or in the interface of dimerization. From this information, we are able to compute the conservation of any residue in the binding site / interface of dimerization. Here, we give an illustration in the case of the residue $124$, which have a high degree of conservation among all the studied GSTs and have been identified as a part of the binding site in $\%$.

\begin{figure}[h!]
	\label{MSA + AF + AFi}
	\raggedright\textbf{A}
	\includegraphics[width = .99\linewidth]{figures/MSA_matrix.pdf}\\[.5cm] % MSA + AF + AFi
	\begin{minipage}{.75\linewidth}
		\textbf{B}\\
		\includegraphics[height = 2cm]{figures/MSA_Proba.jpg} 
	\end{minipage}
	\begin{minipage}{.22\linewidth}
		\textbf{C}\\
		\includegraphics[height = 2cm]{figures/amino-acid_conservation_BS_j=124.jpg} 
	\end{minipage}
	\caption{Indentification of residues in the binding site and of the interface of dimerization}
\end{figure}
   

\section{Dynamics from Normal Modes}
\noindent As explained in the introduction, this present work not only cares about the informations that have been extracted from the static predictions of the AlphaFold and AlphaFill programs but also about the dynamics of the dimers. In this section we will present the next step of our methodology with the Anisotropic Network Model, starting with the parametrization.

\subsection{Parametrization}
\noindent The parametrization step is needed to make sure that the predictions of the model are physically relevant. From the amino-acid's center of mass, it is very simple to compute the mass-weighted Hessian (see eq. \ref{mass-weighted hessian matrix}). As presented before, a first step is to make sure that the cut-off $R_c$ is such that the eigenvalues of the Hessian are non-null. Taking the exemple of the GSTD1, we computed $\tilde{\omega}_k^2 (R_c)$ for the modes $5$, $6$, $7$ and $8$. 

\begin{figure}[h!]
	\label{Rc param}
	\begin{minipage}{.48\linewidth}
		\textbf{A}\\
		\includegraphics[width = .99\linewidth]{figures/GSTD1_ElasticNetwork.png}
	\end{minipage}	
	\begin{minipage}{.48\linewidth}
		\textbf{B}\\
		\includegraphics[width = .99\linewidth]{figures/GSTD1_ANM-COM_Rc_param.jpg}\\[.5cm]
	\end{minipage}	
	\caption{Coarse grained representation \& cut-off parametrisation of GSTD1 structure}	
\end{figure}

\noindent In the figure \ref{Rc param}, it is clearly visible that for $R_c = 7.5\AA$, the eigenvalues for $k \ge 6$ are no longer nulls. It is then convenient tu use this value of cut-off for this structure. Note that later, such computations will be performed for all 25 structures in order to have a correct representation of the structures' topology. Those eigenvalues are computed for $\gamma = 1$ kcal.mol$^{-1}$.\AA$^{-2}$, but as we have seen before, $\tilde{\omega}_k^2 \propto \gamma$. It is now time to compute the thermal B-factors to be able to compute the optimal $\gamma$ value.

\subsection{Predictions}
\begin{figure}[h!]
	\label{ANM-COM D1}
	\begin{minipage}{.68\linewidth}
		\textbf{A}\\
		\includegraphics[height = 4cm]{figures/GSTD1+GSH_ANM-COM_Bfactors.jpg}
	\end{minipage}
	\begin{minipage}{.30\linewidth}
		\textbf{B}\\
		\includegraphics[height = 4cm]{figures/GSTD1_ANM-COM_Bfactors_structure.png}
	\end{minipage}
	\caption{Thermal B-factors predicted from ANM-COM and comparisions with XRD measurments}
\end{figure}

\section{Dynamics from Molecular Dynamics}

\section{Comparison between Structures}