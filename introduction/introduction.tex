% Abstract :
 

\chapter{Introduction}

\section{Glutathione Transferase}	
Glutathione Transferase (GST) is a superfamily of enzymes that are generally homodimeric structures and multigenic in numerous organisms. They are involved in detoxification process as well as in chemoperception in mammals and insects\cite{F-Neiers-GSTs}. Their main function is to catalyze the conjugation of reduced glutathione (GSH) to xenobiotic electrophilic centers\cite{GST-A1}. In their catalytic cycle, the GSH usually binds in a specific set of amino-acids called G-site and the hydrophobic xenobiotic in the so-called H-site. Interactions between insects and plant’s chemicals lead to a major driving force in herbivorous insect evolution, hence this encourages the study of insect GSTs to understand how spontaneous mutations modify the stability, selectivity and the catalytic efficiencies of this enzyme superfamily. In the present work, we have selected 25 sequences of GST from \textit{drosophilia melanogaster}, $11$ from class $\delta$ and $14$ from class $\varepsilon$.
% picture class
This will constitute a working base for all the incomming analysis.
\section{AlphaFold \& X-ray diffraction experiment}
X-ray diffraction is a powerful experimental technique that have been used extensively to determine the three-dimentional structures of proteins. 
% cite XRD ?
In this technique, a crystal of the protein is bombarded with X-rays, and the resulting diffraction pattern is used to determine the position of atoms within the protein. Over the years, X-ray diffraction experiments have played a pivotal role in determining the structures of tens of thousand of proteins, which are deposited in the Protein Data Bank (PDB). 
% cite PDB ?
However, this process can be time-consuming and technically challenging. Moreover, compared to the vast number of known protein sequences, the ensemble of solved structure is insignificant. In 2021, DeepMind used machine learning approaches with the AlphaFold\cite{AlphaFold} programm. It uses computational models to predict the 3D structures of proteins based on it's sequence with a high accuracy. In the field of de novo design of enzymes, ALphaFold has the potential to revolutionize the way we consider the design process, allowing to predict 3D structures that have not yet been experimentally characterized.\\
\noindent In addition to the initial AlphaFold program, DeepMind developped several other tools that have further expanded the capacities of protein structure predictions. One such tool is AlphaFold-multimer\cite{Multimer}, which allows predictions for the structure of protein complexe such as homodimers. An other one is AlphaFill\cite{AlphaFill}, which predict the positions of ligands, small molecules that bind to protein such as Glutathione. All together, these tools represent a major step forward in the field of protein study and will be at the root of the present work.
\section{Goals}