\chapter{Introduction}
Proteins are biological macromolecules that perform a large variety of functions in living cells comprising biochemical (enzymes), structural, mechanical, and signaling functions. They consist of chains composed of 20 different amino-acids. To perform their functions, proteins interact with small molecules referred to as ligands, which are able to bind to a protein with high affinity and specificity \cite{du2016insights}. These protein/ligand interactions are crucial in biology, particularly in the context of drug design \cite{li2019predicting}. Since proteins interact with a broad range of drugs, it is of particular interest to study the mechanisms of binding of ligands to proteins and its impact on the structural dynamics to gain insights into (i) phenomena involved in the biological process and related to diseases \cite{silva2010ligand} (misfolding, aggregation), and (ii) discovery, design, and development of new drugs \cite{payandeh2021ligand}. The experimental structural data (e.g., X-ray crystallography, NMR, or cryo-EM) provide key structural information of the ligand-bound and ligand-unbound (APO) proteins \cite{chakraborti2021all}. Nevertheless, the static information is not always sufficient for understanding protein–ligand binding mechanisms, especially when pockets are highly flexible and contain several binding sites. Therefore, molecular dynamics (MD) and Normal Mode Analysis (NMA) are powerfull tools that provides a description of the dynamics and structures of protein–ligand systems with a high spatial and temporal resolution.
\section{Glutathione Transferase}
Glutathione transferases (GSTs) belong to a ubiquitous superfamily of enzymes that
metabolize a broad range of reactive toxic compounds by catalyzing the conjugation of reduced tripeptide glutathione ($\gamma$-Glu-Cys-Gly; named GSH) to the electrophilic center of a second substrate \cite{mannervik1985isoenzymes, armstrong1997structure, hayes2005glutathione}, the reactivity of GSH being due to the thiol group SH of the cysteine residue. The conjugation reaction occurs spontaneously but GST accelerates it dramatically. This process of detoxification protects cells against damages caused by both exogenous and endogenous molecules. GSTs were first discovered in liver cells \cite{combes1961liver}, and since then, they have been found to exhibit ligand-binding properties for a large variety of compounds, which are not always their enzymatic substrates \cite{axarli2004characterization}. Therefore, GSTs participate in diverse biological processes, making them multifunctional proteins. Moreover, GSTs are classified into three families according to their location in the cell: cytosolic, mitochondrial, and microsomal, which is not evolutively related to the two other classes \cite{oakley2011glutathione}. First-discovered and most-abundant cytosolic GSTs are divided into 13 classes based on homology of their sequences.Members of the same cytosolic class have at least $40\%$ of sequence identity, while members of different classes must have at most $25\%$ of sequence identity. Even if they present a low homology with the cytosolic GST, mitochondrial GSTs can be considered as a particular class of GSTs (Kappa). Among the 42 GSTs identified in \textit{Drosophila melanogaster}, $\delta$ and $\varepsilon$ are the largest classes, with 25 members \cite{F-Neiers-GSTs}. In their catalytic cycle, the GSH usually binds in a specific set of amino-acids called G-site and the hydrophobic xenobiotic in the so-called H-site. Interactions between insects and plant’s chemicals lead to a major driving force in herbivorous insect evolution, hence this encourages the study of insect GSTs to understand how spontaneous mutations modify the stability, selectivity and the catalytic efficiencies of this enzyme superfamily.
% figure 1 : catalytic cycle of GSTs + evolutionary relationship between GSTs (F. Neiers)
\section{AlphaFold}
X-ray diffraction is a powerful experimental technique that have been used extensively to determine the three-dimentional structures of proteins. 
% cite XRD ?
In this technique, a crystal of the protein is bombarded with X-rays, and the resulting diffraction pattern is used to determine the position of atoms within the protein. Over the years, X-ray diffraction experiments have played a pivotal role in determining the structures of tens of thousand of proteins, which are deposited in the Protein Data Bank (PDB). 
% cite PDB ?
However, this process can be time-consuming and technically challenging. Moreover, compared to the vast number of known protein sequences, the ensemble of solved structure is insignificant. In 2021, DeepMind used machine learning approaches with the AlphaFold\cite{AlphaFold} programm. It uses computational models to predict the 3D structures of proteins based on it's sequence with a high accuracy. In the field of de novo design of enzymes, ALphaFold has the potential to revolutionize the way we consider the design process, allowing to predict 3D structures that have not yet been experimentally characterized.\\
\noindent In addition to the initial AlphaFold program, DeepMind developped several other tools that have further expanded the capacities of protein structure predictions. One such tool is AlphaFold-multimer\cite{Multimer}, which allows predictions for the structure of protein complexe such as homodimers. An other one is AlphaFill\cite{AlphaFill}, which predict the positions of ligands, small molecules that bind to protein such as Glutathione. All together, these tools represent a major step forward in the field of protein study and will be at the root of the present work.
\section{Goals}
