\usepackage[utf8]{inputenc}
\usepackage[T1]{fontenc}
\usepackage{lipsum}% juste utile ici pour générer du faux texte}
\usepackage{mwe}%juste utile ici pour générer de fausses images
\usepackage{amsmath,amsfonts,amssymb}%extensions de l'ams pour les mathématiques
\usepackage{shorttoc}%pour la réalisation d'un sommaire
\usepackage{tikz}
\usepackage{tabularx,color,colortbl}
\definecolor{MPL1}{rgb}{0.89411765, 0.10196078, 0.10980392}
\definecolor{MPL2}{rgb}{0.21568627, 0.49411765, 0.72156863}
\definecolor{MPL3}{rgb}{0.30196078, 0.68627451, 0.29019608}
\definecolor{MPL4}{rgb}{0.59607843, 0.30588235, 0.63921569}
\usepackage{floatrow}
\usepackage{booktabs}
\usepackage{tikz}
\usepackage{tcolorbox}
\newcommand*\circled[1]{\tikz[baseline=(char.base)]{
\node[shape=circle,fill=orange!80!yellow,draw,inner sep=1pt,font=\bfseries] (char) {#1};}}
\floatsetup[table]{capposition=top}
\usepackage[first=0,last=9]{lcg}
\newcommand{\ra}{\rand0.\arabic{rand}}
\usepackage{graphicx}%pour insérer images et pdf entre autres
\graphicspath{{images/}}%pour spécifier le chemin d'accès aux images
\usepackage[left=2.5cm,right=1.5cm,top=2cm,bottom=2cm]{geometry}%réglages des marges du document selon vos préférences ou celles de votre établissement
\usepackage[Bjornstrup]{fncychap}%pour de jolis titres de chapitres voir la doc pour d'autres styles.
\usepackage{fancyhdr}%pour les en-têtes et pieds de pages
\setlength{\headheight}{14.2pt}% hauteur de l'en-tête

%%%%%%%%%%%%%%%%%%%style front%%%%%%%%%%%%%%%%%%%%%%%%%%%%%%%%%%%%%%%%% 
\fancypagestyle{front}{%
	\fancyhf{}%on vide les en-têtes
	\fancyfoot[C]{page \thepage}%
	\renewcommand{\headrulewidth}{0.4pt}%trait horizontal pour l'en-tête
	\renewcommand{\footrulewidth}{0.4pt}%trait horizontal pour les pieds de pages
}
%%%%%%%%%%%%%%%%%%%style main%%%%%%%%%%%%%%%%%%%%%%%%%%%%%%%%%%%%
\fancypagestyle{main}{%
	\fancyhf{}
	%\renewcommand{\CTV}{\fontfamily{phv}\selectfont\Huge \scshape}
	\renewcommand{\chaptermark}[1]{\markboth{\thechapter.\ ##1}{}}
	\renewcommand{\sectionmark}[1]{\markright{\thesection\ ##1}}
	\fancyhead[c]{}
	\fancyhead[RO,LE]{\rightmark}%
	\fancyhead[LO,RE]{}
	\fancyfoot[C]{}
	\fancyfoot[RO,LE]{page \thepage}%
	\fancyfoot[LO,RE]{\leftmark}
}
%%%%%%%%%%%%%%%%%%%style back%%%%%%%%%%%%%%%%%%%%%%%%%%%%%%%%%%%%%%%%%  
\fancypagestyle{back}{%
	\fancyhf{}%on vide les en-têtes
	\fancyfoot[C]{page \thepage}%
	\renewcommand{\headrulewidth}{0.4pt}%trait horizontal pour l'en-tête
	\renewcommand{\footrulewidth}{0.4pt}%trait horizontal pour les pieds de pages
}
%%%%%%%%%%%%%%%%%%%%%%%%%%%%index%%%%%%%%%%%%%%%%%%%%%%%%%%%%%%%%%%%%%%%
\usepackage{makeidx}
\makeindex
\usepackage[english]{babel}%pour un document en français
\usepackage{listings}%pour insérer du code source
\usepackage{hyperref}%rend actif les liens, références croisées, toc…
\hypersetup{colorlinks,%
	citecolor=black,%
	filecolor=black,%
	linkcolor=black,%
	urlcolor=black} 
%%%%%%%%%%%%%%%%%%%%%%%%%%%%biblio%%%%%%%%%%%%%%%%%%%%%%%%%%%%%%%%%%%%%%
\usepackage[style=chem-acs,sorting=none,backend=bibtex]{biblatex}
\AtEveryBibitem{%
	\clearfield{note}%
}
\addbibresource{references/refs.bib}% pour indiquer où se trouve notre .bib
\usepackage{csquotes}% pour la gestion des guillemets français.
%%%%%%%%%%%%%%%%%%%%%%%%%%%%%glossaire%%%%%%%%%%%%%%%%%%%%%%%%%%%%%%%%%%%
\usepackage{glossaries}
\makeglossaries         

\makeatletter
\newenvironment{abstract}{%
	\cleardoublepage
	\null\vfil
	\@beginparpenalty\@lowpenalty
	\begin{center}%
		\bfseries \abstractname
		\@endparpenalty\@M
\end{center}}%
{\par\vfil\null}
\makeatother
%%%%%%%%%%%%%%%%%%%%%%%%%%%%%%%%%%%%%%%%%%%%%%%%%%%%%%%%%%%%%%%%%%%%%%%%%%
\newcommand{\fig}[3]{\begin{figure}[h!]
	\centering
	\includegraphics[width = #2\linewidth]{#1}
	\caption{#3}
\end{figure}}

\usepackage{xcolor}
\newcommand{\red}[1]{{\color{red}#1}}
\newcommand{\blue}[1]{{\color{blue}#1}}
\newcommand{\green}[1]{{\color{green}#1}}

\newcommand{\black}[1]{{\color{black}#1}}

\newcommand{\vv}[1]{\vec{#1}}

% fancy verbatim :
\usepackage{fancyvrb}
\newcommand{\vrb}[1]{\begin{Verbatim}[commandchars=\\\{\}]
#1
\end{Verbatim}}